\section{Introduction}

Modern smartphones have evolved from being mere communications devices into
powerful computing devices that can gather and analyze information about the
user and the current context in (near) real-time. Recent studies have broadened
the scope even further by estimating various psychological factors such
as stress and mood with sensors on a regular mobile phone
\cite{lu12stresssense} \cite{likamwa13moodscope}. This trend is expanding
the discipline of psychophysiology and affective computing
\cite{picard95affective}
to ubiquitous
wearable devices and smartphones.

\textit{Psychological sensing} refers to techniques for identifying
psychological states from sensor data. Examples of psychological states that
can be sensed include personality, mood, stress, anxiety, or tension.
Traditionally investigations of psychological phenomena have been conducted
using direct observations or clinical assessment, specialized wearable sensing
platforms \cite{choudhury03sensing} \cite{olguin09capturing} or instrumented
rooms with high-quality equipment \cite{pianesi08multimodal}. Transferring
these techniques to mobile devices, however, is far from straight-forward.

\section{Related Work}

\subsection{Psychological inference}

The existence of a relationship between various physiological measures and
psychological processes has been recognized for a long time in
psychophysiology. A very early application in this field is the polygraph,
more commonly known as lie detector, that measures several physiological indices
such as blood pressure, heart rate and skin conductivity while the
subject is being interrogated. The idea is that deceptive answers
would trigger a response in the measured indices that a human operator can
distinguish reliably.

While polygraphic lie detection has been a target of considerable criticism,
it is empirically established that arousal and anxiety affect the
physiological indices measured in polygraphy in a predictable manner.
There are in fact a wide range of physiological responses in different
human organ systems due to being innervated by the autonomic nervous system.
More recently, psychophysiology has started also studying responses in
central nervous system with the aid of advanced neuroimaging techniques such
as functional magnetic resonance imaging (fMRI). Examples of these responses
include various kinds of event-related potentials (ERP), brain waves and other
cortical brain potentials.

Some studies have utilized large number of these responses in mobile
inference systems with custom-made complex sensing and inference devices.
Plarre et al. \cite{plarre11continuous} used a custom-made multisensory mobile
device called AutoSense \cite{ertin11autosense} to accurately sense stress
levels by utilizing
electrocardiography (ECG), respiratory inductive plethysmograph (RIP) and
3-axis accelerometer among many other sensors and ground truth stress
assessments by experts and self-reports.

Psychologists have also shown the existence of a correlation between
verbal behaviour and various personality types and mental states.
For example, extraversion
has been shown to be associated with higher and more varied pitch
\cite{scherer78personality}.
Other studies have explored the relationship between
personality types and vocal characteristics even earlier
\cite{addington68relationship}
\cite{aronovitch76voice}.
In fact, voice-based mental state inference is already established enough
that it is being incorporated into
developer-friendly libraries that make it possible to utilize complex
voice inference even in low-performance embedded systems with relative ease.
Chang et al. have described AMMON, a complete emotion recognition
pipeline written in C that also has some support for detecting depression
indicators
\cite{chang11hows}.
The system was evaluated against the SUSAS database (Speech Under Simulated
and Actual Stress) \cite{hansen97getting} and achieved an accuracy of 84.43\%
in recognizing stressed vs. neutral utterances. SUSAS includes annotated
samples recorded in a controlled environment.

Pianesi et al. used annotated audio and video recordings from a simulated
multi-party meeting to infer participants' personality traits
\cite{pianesi08multimodal}.
The task was recorded in a controlled laboratory environment.
Recordings were
segmented into 1-minute-length windows and 22 acoustic features and a number
of visual features were extracted. Each window was classified into one of two
personality class, extraversion (one of the dimensions of Big Five) and locus
of control. The study achieved a classification accuracy of 0.89 and 0.87
for extraversion and locus of control respectively.

\begin{itemize}
  \item Emotion sensing with Deep Learning \cite{han14speech}.
  \item Five emotions
\end{itemize}

\begin{itemize}
  \item cStress by Hovsepian et al. \cite{hovsepian15cstress} tries to
    establish a "gold standard" measure for stress based on electrocardiography
    (ECG) and respiration from inductive plethysmography (RIP) measured with
    AutoSense \cite{ertin11autosense}
  \item The measure is actually a complex SVM classification model turned into
    a probability model with Platt's scaling \cite{platt99probabilistic}
\end{itemize}

\subsection{Mobile applications}

Mobile phone applications that employ psychological sensing and
would be evaluated in the varied and heterogeneous conditions typical of such
systems are still few.

\subsubsection{Audio}

StressSense is a real-time system running on off-the-shelf smartphones developed
by Lu et al. that classifies speech segments
into stressed and neutral classes using a gaussian mixture model and multiple
acoustic features.
\cite{lu12stresssense}
Three different parameterizations were considered, a
universal classifier that is trained for all users, a user-specific classifier
and a hybrid model that starts with the universal model and gradually
gets adjusted as more and more individual information is gathered about the
user. The evaluation was done in multiple trials and under mixed acoustic
environments both indoors and outdoors. The personalized model achieved an
accuracy of 82.9\% for the indoor scenario and 77.9\% for the outdoor scenario.

While sleep quality is not a psychological factor per se, it is found to have
a major impact on general psychological well-being.
Hao et al. presented iSleep, an unobtrusive voice-based sleep quality monitoring
system using smartphones
\cite{hao13isleep}.
It classifies recorded voice frames into events like
coughing, snoring and talking with a decision tree model and associates
sequences of events to various metrics of sleep quality. The system was
evaluated with real world data using actigraphy or movement-based sleep quality
system as the ground truth. The system was found to be robust to different
persons and environmental noises achieving 90\% accuracy for sleep-related
event classification. However, the study also found continuous microphone sampling to
be a major concern for energy consumption and suggests an adaptive sampling
strategy whereby sample rate is increased only when a potentially important
event is detected.

\subsubsection{Application sensors}

Application sensors refer to the use of phone and application usage data
as input features in the inference application. Generally the results have shown
this type of sensor data to provide only some information about
psychological states, instead of being able to rely solely on application
sensor information. A particularly interesting prospect is to use the
relatively energy-efficient phone usage data as support for duty cycling
more energy heavy physical sensors.

Chittaranjan et al. investigated the relationship between rich phone usage data
and self-reported Big-Five personality traits (extraversion, agreeableness,
conscientiousness, emotional stability and openness to experiences
\cite{chittaranjan11mining}
\cite{chittaranjan11whos}. Phone usage
data included anonymized logs of calls, SMS, Bluetooth scans, calling profiles
and application usage. A simple statistical correlation and regression analysis
found several aggregated features that reportedly could be predictive of
the Big-Five personality traits.

More recently, LiKamWa et al. built a mood sensing system that
estimates a linear regression model between phone usage metrics and a
self-reported mood
\cite{likamwa13moodscope} \cite{likamwa11canyour}.
Personalized, one-size-fits-all and hybrid models were evaluated with the
personalized model achieving best accuracy (93\%).

\subsubsection{Motion}

Very few mobile applications have used motion or location data to infer
psychological phenomena. One remarkable study conducted by Gruenerbl et al.
used GPS- and motion sensor -based activity recognition to infer mental and
manic episodes in bipolar patients to aid psychiatric diagnosis
\cite{gruenerbl14using}. In a trial
with 12 bipolar disorder patients over 12 weeks the system achieved mental state
change detection precision and recall of 96\% and 94\% respectively and state
recognition accuracy of 80\%. Ground truth was assessed over a certain
time frame during the study by experienced clinical psychologists. The features
used included the number of distinct locations visited, the number of hours
outdoors, accelerometer average magnitudes and variances and spectral
features among others.

\subsubsection{Holistic}

More holistic approaches that strive to fuse sensor data from multiple different
sensor sources to measure psychological factors have also been developed.
Emerging commercial wearable devices (activity monitors, sleep monitors etc.)
that incorporate sensors more suited to psychophysiological measurements
typically fuse movement, heart beat and even galvanic skin response to provide
the user statistics about sleep quality, mood and other factors.

Chen et al. presented a holistic non-intrusive method for sleep quality
detection on off-the-shelf smartphones
\cite{chen13unobtrusive}.
Their "best effort sleep" model
utilizes various user context measures (prolonged silence, darkness,
motion stationarity) in addition
to phone usage statistics (phone-lock duration, phone-charging, phone-off) as
variables to a regression model that estimates sleep duration. The system
was evaluated with a user study and was found to give acceptable accuracy
when compared to more intrusive methods (sleep wristbands, on-head sensors).

Muaremi et al. developed a highly complex system for evaluating a stress score
based on PANAS (Positive and Negative Affect Schedule) questionnaires presented
regularly to the user, multiple features measurable with
smartphone sensors (acoustic features, physical activity measures) and also
social interaction features measured from phone usage statistics
\cite{muaremi13towards}.
A logit
regression model was fitted to these variables to estimate the stress score
for which a self-assessed reference value was acquired through questionnaires.

\subsection{Mobile devices in psychological research}

A general off-the-shelf smartphone with its increasing ubiquity and advancing
sensing features has been noted as being potentially revolutionary as a tool to
conduct large-scale experiments in sociology~\cite{raento09smartphones} and
psychology~\cite{miller12smartphone}. The vision is that custom data collecting
applications with global reach could gather precise, objective, sustained and
ecologically valid data with a much greater efficiency than what was possible
previously, replacing paper-and-pencil surveys, phone surveys and many lab
and field studies.

Rachuri et al. presented EmotionSense, an application framework for conducting
large-scale experiments in social pyschology experiments
\cite{rachuri10emotionsense}.
In essence, the system is a programmable pipeline consisting of sensor monitors,
a knowledge base that stores facts inferred from the sensing data and an
inference engine. The system is rather generic and can be tailored to
support many kinds of data collection experiments.

Smartphones have also been used to implement various ambulatory psychological
monitoring methods such as the electronically activated recorder (EAR)
\cite{mehl12naturalistic}. Also custom-made electronic devices sharing features
with smartphones have been used to infer psychological data. Examples include
AutoSense by Erin et al., a highly sophisticated custom-made electronic sensor
suite collecting ECG, GSR, respiratory rate, temperature and accelerometer
data and communicating with a smartphone via Bluetooth
\cite{ertin11autosense}, and the Mobile Sensing Platform by Choudhury et al.,
an activity recognition platform with rich sensing capabilities predating modern
smartphones \cite{choudhury08msp}.

\begin{itemize}
  \item Deep neural network with a simplified prediction running
    energy-efficiently on a smartphone
    \cite{lane15deepear}
  \item Comparison with EmotionSense \cite{rachuri10emotionsense},
    StressSense \cite{lu12stresssense}, SpeakerSense
    \cite{lu11speakersense} ja JigSaw (Ambient scene detection)
    \cite{lu10jigsaw}. Claims to beat them all.
  \item Emotion and stress sound data from the Emotional Prosody Speech and
    Transcripts library \cite{??emotional}
\end{itemize}

\section{Methodology}

\section{Evaluation}

\begin{itemize}
  \item StressSense \cite{lu12stresssense} used binary classification and
    collected data with experiments that subjected participants to known
    stressors.  Effects were "verified" with galvanic skin response (GSR)
    measurements.
  \item The two stressing situations were a job interview (indoors) and a
    marketing job involving recruiting new participants to scientific studies
    (outdoors). Also two neutral tasks were conducted, indoors and outdoors.
\end{itemize}

\begin{itemize}
  \item cStress \cite{hovsepian15cstress} used a procedure adopted from
    \cite{plarre11continuous} where subjects were exposed to three
    stressors: public speaking, mental arithmetic in sitting and standing
    positions and exposure to cold water :) A 30-rest was used as a baseline.
\end{itemize}
